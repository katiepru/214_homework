\documentclass[letterpaper,12pt]{article}

\usepackage[margin=.5in]{geometry}

\begin{document}

\noindent Joshua Matthews \& Kaitlin Poskaitis

\noindent Systems Programming

\noindent String Tokenizer

\section{Design}
When invoked with two string pointers as input, TKCreate tokenizes the second
string by replacing any characters present in the first string with null bytes.
The result is then stored in a struct along with and index and total string
length, so that TKGetNextToken knows where to start printing from and when it
has truly reached the end of the string.

\subsection{Special Characters}
Before tokenizing both strings are preprocessed and any special
backslash-escaped characters are replaced by their corresponding ASCII values.

\subsection{Checking Delimiters}
Rather than check each character in the delimiters string for a match to each
character in the string to be tokenized, an array of booleans (really just
chars) of length 128 is created, and any characters in the delimiters string has
it's corresponding ASCII value in the array set to TRUE. This is then used to
check if a given character in the tokenized string should act as a delimiter.

\section{Efficiency}
\subsection{TKCreate}
Let $n$ equal the combined length of both strings. The tokenizer first processes
both strings to replace special characters, adding up to one $O(n)$ operation.
It then maps the delimiters to the array of ASCII booleans and uses that to
tokenize the second string. Since the array makes checking if a character is a
delimiter a constant-time operation, this adds up to another $O(n)$ operation.
The final efficiency is therefore $O(n + n) = O(2n) = O(n)$.

\subsection{TKGetNextToken}
First checks that we have not already reached the end of the string, and if not
then copies and returns the string pointed to by the current index, changes that
index to just past the terminating null byte to point at the next token. The
string copy operation is dependent on the length of the string to copied, so
efficiency is $O(n)$.

\subsection{TKDestroy}
First frees the malloced memory storing the delimited string, and then the
encompassing Struct. Both are constant-time operations, efficiency is $O(1)$.

\end{document}
