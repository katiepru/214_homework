\documentclass[letterpaper,12pt]{article}

\usepackage[margin=.5in]{geometry}

\begin{document}

\noindent Joshua Matthews \& Kaitlin Poskaitis

\noindent Systems Programming

\noindent Book Order

\section{Design}
This program processes a series of book orders in multiple threads. When run it
creates a single producer thread to generate the data, and several consumer
threads to process it. The specifics of each step are described below.

\subsection{Customer and Order Databases}
The list of customers and their information is read from a file and stored in a
prefix tree before any threads are created. The struct used to store each
customers information includes a mutex prevent race conditions, and a queue to
store their transaction history. The orders are also stored in a prefix tree
containing synchronous queues. This tree is also built before spawning threads,
but the queues are initially left empty. They are later populated by the
producer thread.

\subsection{Synchronous Queues}
The synchronous queues are constructed using an internal linked lists, and each
queue has a mutex and semaphore. The mutex is used to lock the structure when
enqueuing or dequeuing, and the semaphore acts as a counter to indicate when
there are enqueued items waiting to be processed.

\subsection{Producer Thread}
The producer tread reads in orders from a file, and enqueues each one into the
appropriate category in the categories prefix tree. Upon enqueueing an item, it
posts to that queue's semaphore to signal it's consumer thread to process the
data. Once all orders have been processed and enqueued, NULL is enqueued into
each queue, which signals all the consumer threads to shut down.

\subsection{Consumer Threads}
A consumer thread is spawned for each book category given in the input. Each
consumer thread watches the queue associated with it's category and waits on
its semaphore. When the producer thread posts to the semaphore, this signals
the producer thread to process the incoming data, after which it dequeues and
processes the new data. If a value of NULL is dequeued the thread exits, since
this value acts as a sentinel to signal that all orders have been processed.

\subsection{Final Output}
Once the producer and all consumer threads exit, each customers order history is
printed out. As this is done a total revenue is calculated from all successful
transactions, and printed at the end.

\section{Efficiency}
Building the customer and order databases is accomplished in O(n) by using
prefix trees. Enqueueing and dequeuing also takes O(n). Finally, signalling each
thread to exit and outputting each customers transaction history is done through
a DFS on each prefix tree, which is also O(n).

\end{document}
