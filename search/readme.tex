\documentclass[letterpaper,12pt]{article}

\usepackage[margin=.5in]{geometry}

\begin{document}

\noindent Joshua Matthews \& Kaitlin Poskaitis

\noindent Systems Programming

\noindent Search

\section*{Design and Time Efficiency}
The indexer remains largely unchanged from the previous assignment.  It still
outputs the inverted index file in the same format and has the same efficiency.

Essentially our search program works by preprocessing the inverted index before
any queries are entered. During this preprocessing step, it rebuilds a prefix
tree of terms with sorted lists of the files that contain each term attached to
each tree node that marks the end of a term.  These lists are sorted in reverse
alphabetical order.

Once a query is entered, this tree is searched in order to determine the output.
If it is an or query, the programs searches for each term and adds all of the
files to an output sorted list, which is sorted in alphabetical order to allow
for constant inserts.  Any duplicates are ignored in this process by the sorted
list implementation.  If we let $n$ be the number of terms queried for, $m$ be the
number of files they appear in in total, and $k$ be the length of each word, the
overall efficiency is $O(n(m+k))$.  This is because the tree much be traversed for
each word $(nk)$, and then each file must be added to the output list $(nm)$.

For the and query, each term is found in the sorted list and an iterator is
attached to each sorted list. For each file in the first list, if it exists in
every other list, then it is added to an output list sorted alphabetically. This
efficiency is also $O(n(k+m))$ because first all of the terms must be found in
the tree $(nk)$.  Then, doing the comparisons for each iterator would take
$(nm)$.

This leaves us with a total efficiency of $O(n(k+m))$ for both preprocessing and
queries.

\section*{Exceptional cases}
\begin{itemize}
    \item {\bf File does not exist or is not readable}: In this case the program
    will report that it could not open the file and exit with a nonzero return
    value.
    \item {\bf Term in query does not exist in any files}: This will not affect
    the program. If it is an or query, then the contents of the rest of the
    terms will be printed. If it is an and query, then the result will be empty.
    \item {\bf Invalid query operator}: Program will report that it was an
    incorrect query operator and prompt for a new query.
    \item {\bf Query q is followed by terms}: Program terminates and terms are
    ignored.
\end{itemize}

\section*{Memory usage}
The preprocessing is where most of the memory is used. The file is read in word
by word, so that needs to be stored.  These word are then inserted into the
prefix tree, which stores the character, a pointer to the parent, a pointer to
the sorted list of file names, and an array of pointers to its 36 children, most
of which will be null.  The complexity of the prefix tree would be approximately
$O(n)$ where $n$ is the total number of characters in all of the words.  The
sorted lists used would be approximately $O(m)$ where $m$ is the number of
characters in the filenames.


\end{document}
