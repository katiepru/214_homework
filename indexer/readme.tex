\documentclass[letterpaper,12pt]{article}

\usepackage[margin=.5in]{geometry}

\begin{document}

\noindent Joshua Matthews \& Kaitlin Poskaitis

\noindent Systems Programming

\noindent Indexer

\section*{Design}
This program utilizes prefix trees and sorted lists in its implementation.
First, a file from the directory gets tokenized and all of its tokens get added
to a temporary prefix tree along with their frequencies. Then, the information
from this tree gets added into a master prefix tree, and the temporary tree gets
destroyed.  The filenames and frequencies are stored in a sorted list that can be
found at each complete word in the prefix tree.  This information is then
written to the output file. If the file does not exist, we create it.

Notes\\
If the output file already exists, it appends to it. If it does not, it creates
it.\\
If the directory is empty for does not exist, output file is written was blank.

\section*{Efficiency}
In order to build or dfs on a prefix tree, the efficiency is O(nk), where n is
the number of words and k is the average word length.  We create m+1 prefix
trees, where m is the number of files, but each file only contains a subset of
the total words. In total, the efficiency should be O(mnk).

\end{document}
